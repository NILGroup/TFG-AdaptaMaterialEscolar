  
%---------------------------------------------------------------------%
%																	  %
%               Cap�tulo 6 - Trabajo individual				  		  %
%																	  %
%---------------------------------------------------------------------%
\setlength{\parskip}{\baselineskip}

\chapter{Trabajo Individual}

\begin{resumen}
	En este cap�tulo se habla del trabajo individual que ha realizado cada miembro del equipo en el proyecto.	
\end{resumen}
	
%------Primera secci�n: Pablo------%
\section{Pablo}
%-----------------------------------%
\label{cap6:sec:pablo}

	
	
%------Segunda secci�n: Natalia------%
\section{Natalia}
%-----------------------------------%
\label{cap6:sec:natalia}

Una de las tareas que realic� fue actuar como punto de comunicaci�n con las profesoras del IES Maestro Juan de �vila de Ciudad Real. 

Contact� con ellas para realizar una entrevista presencial, donde se les pidi� colaboraci�n en el proyecto, de forma que pudieran aportar su punto de vista profesional y experiencia trabajando con alumnos con necesidades especiales.

A partir de la reuni�n, correspondiente a la primera iteraci�n, realic� la captaci�n de requisitos inicial del proyecto en base a las necesidades que para las profesoras del Aula TEA a�n no estaban cubiertas por ninguna herramienta de trabajo que pudiera facilitar la adaptaci�n de material escolar, aunque tampoco para ning�n otro docente de ese centro.

Los requisitos finales obtenidos se divid�an en listados de tareas, diferenciados en  adaptaciones de formato, actividades y/o ex�menes y temario.

Evalu� los requisitos que se hab�an obtenido como resultado de la captaci�n de requisitos de forma individual, seg�n los conocimientos adquiridos durante el grado, y calcul� la prioridad final de cada tarea con la puntuaci�n que hab�an asignado las profesoras y todos los componentes del equipo de desarrollo.

Realic� en com�n con mis compa�eros de equipo en el proyecto un dise�o para la interfaz gr�fica de la aplicaci�n web para conseguir un prototipo inicial, de forma que vimos las diferentes herramientas que pod�amos a�adir al trabajo para aportar valor, como por ejemplo, un editor de c�digo abierto de tipo WYSIWYG (What You See Is What You Get).

Tambi�n realic� un dise�o distinto al prototipo inicial del proyecto de forma individual, para que no hubiera influencia entre el equipo y se pudiera debatir luego sobre los cambios que cada uno consideraba importantes, de manera que fuera posible mejorar el dise�o inicial a�adiendo los cambios que a todos nos hab�an parecido necesarios y relevantes.

Investigu� sobre c�mo subir un fichero de texto que estuviera almacenado en una ubicaci�n local del ordenador y sobre el editor de texto CKEditor 5, de tipo WYSYWYG, para poder a�adirlo al prototipo individual y comprobar la dificultad inicial de implementar los requisitos necesarios para el proyecto.


%------�ltima secci�n: Jorge------%
\section{Jorge}
%-----------------------------------%
\label{cap6:sec:jorge}

	

% Variable local para emacs, para  que encuentre el fichero maestro de
% compilaci�n y funcionen mejor algunas teclas r�pidas de AucTeX
%%%
%%% Local Variables:
%%% mode: latex
%%% TeX-master: "../Tesis.tex"
%%% End: