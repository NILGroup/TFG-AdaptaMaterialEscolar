  
%---------------------------------------------------------------------%
%																	  %
%               Capítulo 6 - Trabajo individual				  		  %
%																	  %
%---------------------------------------------------------------------%
\setlength{\parskip}{\baselineskip}

\chapter{Trabajo Individual}

\begin{resumen}
	En este capítulo se habla del trabajo individual que ha realizado cada miembro del equipo en el proyecto.	
\end{resumen}
	
%------Primera sección: Pablo------%
\section{Pablo}
%-----------------------------------%
\label{cap6:sec:pablo}

	
	
%------Segunda sección: Natalia------%
\section{Natalia}
%-----------------------------------%
\label{cap6:sec:natalia}
Una de las tareas que realicé fue actuar como punto de comunicación con las profesoras del IES Maestro Juan de Ávila de Ciudad Real. 

Contacté con ellas para realizar una entrevista presencial, donde se les pidió colaboración en el proyecto, de forma que pudieran aportar su punto de vista profesional y experiencia trabajando con alumnos con necesidades especiales.

A partir de la reunión, correspondiente a la primera iteración, realicé la captación de requisitos inicial del proyecto en base a las necesidades que para las profesoras del Aula TEA aún no estaban cubiertas por ninguna herramienta de trabajo que pudiera facilitar la adaptación de material escolar, aunque tampoco para ningún otro docente de ese centro.

Los requisitos finales obtenidos se dividían en listados de tareas, diferenciados en  adaptaciones de formato, actividades y/o exámenes y temario.

Evalué los requisitos que se habían obtenido como resultado de la captación de requisitos de forma individual, según los conocimientos adquiridos durante el grado, y calculé la prioridad final de cada tarea con la puntuación que habían asignado las profesoras y todos los componentes del equipo de desarrollo.

Realicé en común con mis compañeros de equipo en el proyecto un diseño para la interfaz gráfica de la aplicación web para conseguir un prototipo inicial, de forma que vimos las diferentes herramientas que podíamos añadir al trabajo para aportar valor, como por ejemplo, un editor de código abierto de tipo WYSIWYG (What You See Is What You Get).

También realicé un diseño distinto al prototipo inicial del proyecto de forma individual, para que no hubiera influencia entre el equipo y se pudiera debatir luego sobre los cambios que cada uno consideraba importantes, de manera que fuera posible mejorar el diseño inicial añadiendo los cambios que a todos nos habían parecido necesarios y relevantes.

Investigué sobre cómo subir un fichero de texto que estuviera almacenado en una ubicación local del ordenador y sobre el editor de texto CKEditor 5, de tipo WYSYWYG, para poder añadirlo al prototipo individual y comprobar la dificultad inicial de implementar los requisitos necesarios para el proyecto.
	

%------Última sección: Jorge------%
\section{Jorge}
%-----------------------------------%
\label{cap6:sec:jorge}

	

% Variable local para emacs, para  que encuentre el fichero maestro de
% compilación y funcionen mejor algunas teclas rápidas de AucTeX
%%%
%%% Local Variables:
%%% mode: latex
%%% TeX-master: "../Tesis.tex"
%%% End: