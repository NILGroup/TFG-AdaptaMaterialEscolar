  
%---------------------------------------------------------------------%
%																	  %
%               Cap�tulo 7 - Implementaci�n				  		  %
%																	  %
%---------------------------------------------------------------------%
\setlength{\parskip}{\baselineskip}

\chapter{Implementaci�n}

\begin{resumen}
		En este cap�tulo se explica c�mo se ha implementado la aplicaci�n de AdaptaMaterialEscolar, los patrones de dise�o utilizados para su desarrollo y la estructura del c�digo.
\end{resumen}


\section{Frontend}
%-----------------------------------%
\label{cap7:sec:frontend}

\section{Redux}
%-----------------------------------%
\label{cap7:sec:redux}	
	
	
\section{Estructura del c�digo}
%-----------------------------------%
\label{cap7:sec:estructura}

La organizaci�n de directorios del proyecto gira en torno al patr�n de dise�o Redux, como se ha explicado en el apartado \ref{cap7:sec:redux} de este mismo cap�tulo.

El c�digo de la funcionalidad de la aplicaci�n se estructura principalmente en cuatro carpetas: components, pages, ckeditor y redux; adem�s, contamos con el fichero App, que supone el punto de inicio del programa.

El contenido de cada carpeta es el siguiente:

\begin{itemize}
	\item \textbf{Components:} Representado en la Figura \ref{fig:srcComponents}, contiene cada componente que representa una funcionalidad para la aplicaci�n, o parte de ella, y todos cuentan, como m�nimo, con los siguientes ficheros:
		\begin{itemize}
		\item Fichero modal de la funcionalidad del componente.
		\item Fichero que realiza las modificaciones pertinentes para ejecutar la funcionalidad.
		\item Fichero de estilo del componente.
		\end{itemize}
\end{itemize}

Uno de los componentes m�s importantes es el de la barra de herramientas (toolbar), que sirve como hilo conductor de la selecci�n de los diferentes tipos de adaptaciones que pueden agregarse al documento, mediante el patr�n de dise�o del despachador (dispatcher).

\figura{BitMap/srcComponents}{width=0.3\textwidth}{fig:srcComponents}{Ejemplo de organizaci�n de Components}

\begin{itemize}
	\item \textbf{Pages:} Representado en la Figura \ref{fig:srcPages}, contiene la p�gina del editor, que permite cargar un fichero inicial para comenzar la adaptaci�n, y la p�gina de ayuda, que muestra informaci�n relevante sobre la aplicaci�n y su uso. En los dos casos se cuenta con:
		\begin{itemize}
		\item Fichero de funcionalidad de la p�gina.
		\item Fichero de estilo de la p�gina.
		\end{itemize}
\end{itemize}

\figura{BitMap/srcPages}{width=0.25\textwidth}{fig:srcPages}{Ejemplo de organizaci�n de Pages}

\begin{itemize}
	\item \textbf{Redux:} Representado en la Figura \ref{fig:srcRedux}, contiene la implementaci�n del patr�n de dise�o que se ha aplicado a toda la aplicaci�n, basado en el almacenamiento de los datos que necesita cada componente. La carpeta consta del store y el reducer, elementos necesarios para el desarrollo del mismo. Adem�s, se incluye tambi�n cada componente principal con los siguientes ficheros:
		\begin{itemize}
		\item Fichero action: controla las diferentes acciones que puede ejecutar cada componente.
		\item Fichero reducer: controla el estado de los componentes seg�n el tipo de acci�n que realicen, partiendo de un estado inicial.
		\item Fichero selectors: devuelve el valor de los atributos de cada componente.
		\item Fichero types: contiene un listado de acciones posibles para cada componente. 
		\end{itemize}
\end{itemize}

\figura{BitMap/srcRedux}{width=0.5\textwidth}{fig:srcRedux}{Ejemplo de organizaci�n de Redux}

\begin{itemize}
	\item \textbf{CKEditor:} Representado en la Figura \ref{fig:srcCkeditor}X,contiene todos componentes software o plugins que se han desarrollado para la aplicaci�n, permitiendo la interacci�n a partir de comandos que controlan las diferentes funcionalidades. Consta de un fichero que devuelve la instancia del editor CKeditor5 y una carpeta para cada plugin, con los siguientes archivos:
		\begin{itemize}
		\item Fichero de implementaci�n del plugin.
		\item Fichero de inserci�n del plugin mediante el patr�n de dise�o de comando (command).
		\end{itemize}
\end{itemize}

\figura{BitMap/srcCkeditor}{width=0.5\textwidth}{fig:srcCkeditor}{Ejemplo de organizaci�n de CKEditor}

% Variable local para emacs, para  que encuentre el fichero maestro de
% compilaci�n y funcionen mejor algunas teclas r�pidas de AucTeX
%%%
%%% Local Variables:
%%% mode: latex
%%% TeX-master: "../Tesis.tex"
%%% End: