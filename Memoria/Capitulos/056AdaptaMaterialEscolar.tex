%------Primera secci�n: AdaptaMaterialEscolar------%
\section{AdaptaMaterialEscolar}
%--------------------------------------------------%
\label{cap6:sec:adaptamaterialescolar}
	
	AdaptaMaterialEscolar es una aplicaci�n web creada con React (ver secci�n \ref{cap3:sec:react}), cuyo c�digo se puede encontrar en repositorio del grupo NIL en GitHub\footnote{\url{https://github.com/NILGroup/TFG-AdaptaMaterialEscolar/tree/master/Codigo}}. La web es accesible a trav�s de la siguiente url: \url{https://holstein.fdi.ucm.es/tfg/2021/adapta/}.
	La aplicaci�n est� formada por una serie de componentes, los cuales se contar�n con m�s detalle en la secci�n \ref{cap6:sec:componentes}. Las adaptaciones y ejercicios de la aplicaci�n se encuentran en un componente llamado Toolbar, situado en la parte superior del editor, el cual tiene los siguientes elementos:
	\begin{itemize}
		\item Pictogramas. Este componente busca pictogramas correspondientes al texto introducido. Haciendo clic en un pictograma hace que �ste se lleve al editor.
		\item Definiciones. Este componente crea una lista de definiciones (y/o preguntas), adem�s de establecer el n�mero de l�neas para contestarlas. Tambi�n tiene una opci�n para a�adir un espaciado extra entre l�neas para usuarios que tengan una letra m�s grande.
		\item Sopa de letras. Este componente crea una sopa de letras dando una serie de opciones (explicadas en la secci�n \ref{cap3:sec:wordsearch}), tales como el n�mero de filas, columnas, direcciones cardinales para buscar la palabra...
		\item Verdadero y falso. Este componente crea una lista de oraciones y a�ade al final una l�nea para contestar dicha oraci�n.
		\item Desarrollo. Este componente crea un ejercicio de desarrollo, insertando un enunciado y el n�mero de l�neas para desarrollarlo. Tambi�n tiene una opci�n para a�adir un espaciado extra entre l�neas, al igual que en el ejercicio de definiciones.
	\end{itemize}
	
	