%---------------------------------------------------------------------
%
%                          Cap�tulo 2
%
%---------------------------------------------------------------------
\nocite{*}
\chapter{Introduction}

\begin{resumen}
This chapter introduces the Final Degree Project that will be presented in this document. First, in Section \ref{cap2:motivation}, the motivation that lead to this project will be explained. Then, in Section \ref{cap2:objectives}, the objective to be achieved. Finally, the structure of the final project, in Section \ref{cap2:structure}.
\end{resumen}


\section{Motivation}
\label{cap2:motivation}

	The purpose of school education is to promote the development of certain skills and the learning of certain contents necessary for students to become active members of society. To achieve this, the school must offer an educational response that avoids discrimination and ensures equal opportunities.

	Currently, the Spanish education system is organized into eight stages or levels that guarantee the right to inclusive education for students at every stage of their cognitive and emotional development; of these, only two are compulsory: Primary Education (PE) and Compulsory Secondary Education (ESO). 

	In Spain there are approximately 2 million students with special educational needs (SEN), although not all of them have some type of disability, compared to the 8 million students in non-university general education.

	The materials, resources and contents presented and used in educational centers are regulated by law in the educational or school curriculum. The educational curriculum tries to guarantee that all students finish each year equally prepared. The educational curriculum is the regulation of the elements that determine the teaching and learning processes of each of the subjects and includes: the objectives of each teaching and educational stage, the competences and contents, the didactic methodology, the learning standards and results and the evaluation criteria.

	In the school curriculum there are common educational needs, shared by all the students. However, not all students face learning with the same tools; each student has an individual need. Most of the individual needs of students are solved through ``simple'' actions: giving the student more time to learn certain contents, designing complementary activities... However, there are individual needs that cannot be solved by these means, requiring a series of special pedagogical measures different from those usually required by most students. In this case we speak of special educational needs, and to meet these needs curricular adaptations are necessary. There are two types of curricular adaptations:

\begin{itemize}
	\item \textbf{Non-significant adaptation}: the curricular contents of the subjects are not modified, but other resources like the materials or exams are adapted. The teachers will be in charge of making these adaptations.
	\item \textbf{Significant adaptation}: sections of the official curriculum are eliminated.
\end{itemize}

	Teachers spend too much time creating academic material for students who need non-significant adaptations. Among other things, they have to adjust the text font and size, search for images on the Internet or scan them from books, write summaries highlighting the most relevant information, etc. This FDP aims to provide a tool for teachers to facilitate the adaptation of the curricular contents of the subjects, thus reducing the time and effort that teachers must devote to these tasks.

\section{Objetives}
\label{cap2:objectives}

	The objective of this FDP is to develop a working tool for teachers that allows them to adapt the resources of the subjects in any format in an intuitive, easy and fast way, and thus be able to create customized didactic units that fit the needs of each student.
	
	The tool resulting from this FDP will allow the creation of personalized school material tailored to each student, allowing format changes, the creation of different types of exercises (such as fill-in-the-blank exercises, word searches or development exercises) and an integrated pictogram search engine.

	The tool will be based on a Redux Architecture, in which the views of the data models will be separated internally. In addition, external services will be used to provide functionality to our application.

	Also, in order to adapt the application to the real needs of our end users (teachers), we will follow a User-Centered Design (UCD). For this, we will conduct interviews with end users in order to define their needs, which design best suits their needs and which functionalities have the highest priority. We will also count on them to evaluate the tool.

	As for the academic objectives, our main goal is to apply, in a real project, the knowledge acquired during the Software Engineering Degree and to extend it.


\section{Project structure}
\label{cap2:structure}
The structure of the project consists of ten chapters, including this introductory chapter, which explains the motivation and objective of the work. The following is a summary of each chapter:

\begin{itemize}
	\item In \textbf{chapter three}, the state of the art is presented, where the domain in which the project is framed will be explained.
	\item In \textbf{chapter four} the tools used in the development of the project are described.
	\item \textbf{Chapter five} explains the development methodology that has been used during the whole process and the reason why this methodology was chosen.
	\item In \textbf{chapter six}, the phase of requirements capture, design and implementation of the application is explained.
	\item In \textbf{chapter seven} it is shown how the evaluation of AdaptaMaterialEscolar has been done.
	\item In \textbf{chapter eight}, the conclusions and future work are described.
	\item In \textbf{chapter nine} the same conclusions and future work are described, translated into English.
	\item In \textbf{chapter ten} the individual work that each team member has done on the project is presented.
\end{itemize}


% Variable local para emacs, para  que encuentre el fichero maestro de
% compilaci�n y funcionen mejor algunas teclas r�pidas de AucTeX
%%%
%%% Local Variables:
%%% mode: latex
%%% TeX-master: "../Tesis.tex"
%%% End: