%---------------------------------------------------------------------
%
%                          config.tex
%
%---------------------------------------------------------------------
%
% Contiene la  definici\'on de constantes  que determinan el modo  en el
% que se compilar\'a el documento.
%
%---------------------------------------------------------------------
%
% En concreto, podemos  indicar si queremos "modo release",  en el que
% no  aparecer\'an  los  comentarios  (creados  mediante  \com{Texto}  o
% \comp{Texto}) ni los "por  hacer" (creados mediante \todo{Texto}), y
% s\'i aparecer\'an los \'indices. El modo "debug" (o mejor dicho en modo no
% "release" muestra los \'indices  (construirlos lleva tiempo y son poco
% \'utiles  salvo  para   la  versi\'on  final),  pero  s\'i   el  resto  de
% anotaciones.
%
% Si se compila con LaTeX (no  con pdflatex) en modo Debug, tambi\'en se
% muestran en una esquina de cada p\'agina las entradas (en el \'indice de
% palabras) que referencian  a dicha p\'agina (consulta TeXiS_pream.tex,
% en la parte referente a show).
%
% El soporte para  el \'indice de palabras en  TeXiS es embrionario, por
% lo  que no  asumas que  esto funcionar\'a  correctamente.  Consulta la
% documentaci\'on al respecto en TeXiS_pream.tex.
%
%
% Tambi\'en  aqu\'i configuramos  si queremos  o  no que  se incluyan  los
% acr\'onimos  en el  documento final  en la  versi\'on release.  Para eso
% define (o no) la constante \acronimosEnRelease.
%
% Utilizando \compilaCapitulo{nombre}  podemos tambi\'en especificar qu\'e
% cap\'itulo(s) queremos que se compilen. Si no se pone nada, se compila
% el documento  completo.  Si se pone, por  ejemplo, 01Introduccion se
% compilar\'a \'unicamente el fichero Capitulos/01Introduccion.tex
%
% Para compilar varios  cap\'itulos, se separan sus nombres  con comas y
% no se ponen espacios de separaci\'on.
%
% En realidad  la macro \compilaCapitulo  est\'a definida en  el fichero
% principal tesis.tex.
%
%---------------------------------------------------------------------


% Comentar la l\'inea si no se compila en modo release.
% TeXiS har\'a el resto.
% ���Si cambias esto, haz un make clean antes de recompilar!!!
\def\release{1}


% Descomentar la linea si se quieren incluir los
% acr\'onimos en modo release (en modo debug
% no se incluir\'an nunca).
% ���Si cambias esto, haz un make clean antes de recompilar!!!
%\def\acronimosEnRelease{1}


% Descomentar la l\'inea para establecer el cap\'itulo que queremos
% compilar

% \compilaCapitulo{01Introduccion}
% \compilaCapitulo{02EstructuraYGeneracion}
% \compilaCapitulo{03Edicion}
% \compilaCapitulo{04Imagenes}
% \compilaCapitulo{05Bibliografia}
% \compilaCapitulo{06Makefile}

% \compilaApendice{01AsiSeHizo}

% Variable local para emacs, para  que encuentre el fichero maestro de
% compilaci\'on y funcionen mejor algunas teclas r\'apidas de AucTeX
%%%
%%% Local Variables:
%%% mode: latex
%%% TeX-master: "./Tesis.tex"
%%% End:
