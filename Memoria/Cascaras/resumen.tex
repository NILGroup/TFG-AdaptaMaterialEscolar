%---------------------------------------------------------------------
%
%                      resumen.tex
%
%---------------------------------------------------------------------
%
% Contiene el cap�tulo del resumen.
%
% Se crea como un cap�tulo sin numeraci�n.
%
%---------------------------------------------------------------------

\chapter{Resumen}
\cabeceraEspecial{Resumen}

\begin{FraseCelebre}
\begin{Frase}

\end{Frase}
\begin{Fuente}

\end{Fuente}
\end{FraseCelebre}

En la actualidad la educaci�n tiene la finalidad de promover el aprendizaje y desarrollo de las capacidades de sus alumnos, incluyendo a aquellos con necesidades educativas especiales, sin que ello suponga que tengan una discapacidad. Por ley, en el curr�culum educativo o escolar se tratan todos los materiales, recursos y contenidos que se debe garantizar en el proceso de formaci�n del alumnado durante su recorrido acad�mico, en igualdad de condiciones; sin embargo, debido a que no todos los estudiantes cuentan con las mismas herramientas de aprendizaje, existe la posibilidad de realizar adaptaciones curriculares significativas y no significativas. El proyecto consiste en el desarrollo de una aplicaci�n creada espec�ficamente para los docentes, para facilitarles la redacci�n de ex�menes, actividades y temario personalizado para cada individuo, de forma f�cil y r�pida, de manera que, adem�s, se pueda estandarizar un formato com�n de material escolar para todos los alumnos que necesiten alg�n tipo de adaptaci�n. De �sta forma, pretendemos desarrollar AdaptaMaterialEscolar para cubrir una necesidad real que existe en el �mbito docente.

\endinput
% Variable local para emacs, para  que encuentre el fichero maestro de
% compilaci�n y funcionen mejor algunas teclas r�pidas de AucTeX
%%%
%%% Local Variables:
%%% mode: latex
%%% TeX-master: "../Tesis.tex"
%%% End:
