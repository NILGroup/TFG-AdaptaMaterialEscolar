%---------------------------------------------------------------------
%
%                      resumen.tex
%
%---------------------------------------------------------------------
%
% Contiene el cap�tulo del resumen.
%
% Se crea como un cap�tulo sin numeraci�n.
%
%---------------------------------------------------------------------

\chapter{Resumen}
\cabeceraEspecial{Resumen}

\begin{FraseCelebre}
\begin{Frase}

\end{Frase}
\begin{Fuente}

\end{Fuente}
\end{FraseCelebre}

Por ley, en el curr�culum educativo o escolar trata de garantizar que el proceso de formaci�n del alumnado cuente con los materiales, recursos y contenidos necesarios durante su recorrido acad�mico, en igualdad de condiciones. Sin embargo, debido a que no todos los estudiantes cuentan con las mismas herramientas de aprendizaje, existe la posibilidad de realizar adaptaciones curriculares significativas y no significativas. 


El proyecto consiste en el desarrollo de una aplicaci�n creada espec�ficamente para facilitar a los docentes la adaptaci�n curricular no significativa de ex�menes, actividades y unidades did�cticas personalizadas para cada alumno. Para ello, hemos implementado un editor de texto que cuenta los formatos y las funcionalidades m�s utilizadas por los profesores, con el fin de que puedan realizar las adaptaciones de manera m�s r�pida, siguiendo un Dise�o Centrado en el Usuario. Adem�s, la aplicaci�n web ha sido probada y evaluada por usuarios finales que pretenden usarla en el per�odo escolar. Los resultados obtenidos demuestran que se cubre una necesidad real del profesorado y su inter�s por seguir desarrollando y mejorando AdaptaMaterialEscolar.

\endinput
% Variable local para emacs, para  que encuentre el fichero maestro de
% compilaci�n y funcionen mejor algunas teclas r�pidas de AucTeX
%%%
%%% Local Variables:
%%% mode: latex
%%% TeX-master: "../Tesis.tex"
%%% End:
