%---------------------------------------------------------------------
%
%                      summary.tex
%
%---------------------------------------------------------------------
%
% Contiene el cap�tulo del resumen.
%
% Se crea como un cap�tulo sin numeraci�n.
%
%---------------------------------------------------------------------

\chapter{Summary}
\cabeceraEspecial{Summary}

\begin{FraseCelebre}
\begin{Frase}

\end{Frase}
\begin{Fuente}

\end{Fuente}
\end{FraseCelebre}

Currently, the purpose of education is to promote the learning and development of students' abilities, including those with special educational needs, without implying that they have a disability. By law, the educational or school curriculum deals with all the materials, resources, and contents that must be guaranteed in the training process of students during their academic journey, under equal conditions. However, since not all students have the same learning tools, there is the possibility of making significant and non-significant curricular adaptations. The project consists of the development of an application created specifically for teachers, to facilitate the creation of tests, activities, and didactic units customized for each student, easily and quickly, so that, in addition, a common format of school material can be standardized for all students who need some kind of adaptation. In this way, we intend to develop AdaptaMaterialEscolar to cover a real need that exists in the teaching field.


\endinput
% Variable local para emacs, para  que encuentre el fichero maestro de
% compilaci�n y funcionen mejor algunas teclas r�pidas de AucTeX
%%%
%%% Local Variables:
%%% mode: latex
%%% TeX-master: "../Tesis.tex"
%%% End:
